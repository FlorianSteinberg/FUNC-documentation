\section{A class for Taylor series}

\temp{introduce abbreviations.}

\subsection{The class definition}

\begin{lstlisting}
template<class ARG>
class BASE_ANAL: public FUNC<ARG(ARG)> {
	// members:
		private:
			// the parameters which determine the function:
	  			ps_ptr coeff;
	  			INTEGER k;
	  			INTEGER A;
			// additional data, stored to speed things up.
	  			REAL effective_radius_of_convergence;
			bool has_better_alg = false;
			poly_ptr poly = NULL;
			INTEGER lipschitz;
			// alg_func_ptr<PARAM, RESULT> algorithm = NULL;
	// constructors:
		public:
			BASE_ANAL();
			BASE_ANAL(REAL z);
			BASE_ANAL(
			const ps& _coeff,
			const INTEGER& _k,
			const INTEGER& _A
		);
		BASE_ANAL(const POLY<REAL>&);
	// member functions:
		public:
			POLY<REAL> cut_of_at(const unsigned int&) const;
			ARG operator () (const ARG&) const;
			virtual void derive(unsigned int);
			virtual void anti_derive(unsigned int);
			void has_algorithm(const alg_func<ARG, ARG>&);
			void restore_alg();
			void show_info() const;
		protected:
			virtual void print(iRRAM::orstream&) const;
			virtual void print(int precision) const;
 	// standard operators:
		public:
			BASE_ANAL<ARG>& operator = (const BASE_ANAL<ARG>& f);
			friend BASE_ANAL<ARG>& operator += (const BASE_ANAL<ARG>&);
			...
			friend BASE_ANAL<ARG> operator *<> (
				const BASE_ANAL<ARG>&,
				constBASE_ANAL<ARG>&
			);
			template<class PAR>
			BASE_ANAL<ARG> operator () (const BASE_ANAL<ARG>&);
  	// friends classes:
   		template<class>
   		friend class POLY;
};
\end{lstlisting}

\subsection{The evaluation}
